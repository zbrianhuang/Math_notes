\documentclass{article}
\usepackage{mathtools}
\title{Chapter 6}
\author{Brian Huang}
\date{December 4, 2023}
\begin{document}
   \maketitle

   \newpage
   \section*{6.7 Least-Squares Problem}
    
   \subsection*{Definition}
   If $A$ is $m*n$ matrix and  $b$ is in $\mathbb{R}^m$, a least-squares problem of $Ax=b$ is an $\hat{x}$ in $\mathbb{R}^n$ such that
   \begin{center}
       $\left \| b - A\hat{x} \right \| \leq \left \| b - Ax  \right \|$
   \end{center}
   for all $x$ in $\mathbb{R}^n$ .\\

  \subsection*{Theorem 13}
  The set of least squares solutions of $Ax = b$ coincides with the non-empty set of solutions of the equations:  $A^TAx=A^Tb$ \\

  \subsection*{Theorem 14}
  Let $A$ be an $n*n$ matrix. TFAE:\\
  a. the equation $Ax=b$ has a unique least-squares solution for each $b$ in $\mathbb{R}^n$.\\
  b. The columns of  $A$ are linearly independent.\\
  c. The Matrix $A^TA$ is invertible.\\
  
When these statements are true, the least-squares solution $\hat{x}$ is given by:\\
  
  \begin{center}
      
   $\hat{x} = (A^TA)^{-1}A^Tb$
\end{center} 


\subsection*{Theorem 15}
Given an $m*n$ matrix A with linearly indepedent columns, let $A = QR$ be a  $QR$ factorization of $A$ as in Theorem 12. Then for each $b$ in $\mathbb{R}^n$, the equation $Ax = b $ has a unique least-squares solution given by:
\begin{center}
    $\hat{x} = R^{-1} Q^Tb$
\end{center}


\subsection*{Conclusion}
In summary, The least squares solution aims to find the smallest $\left \|  b-Ax \right \|$.\\
To find it we use the equation from \textbf{Theorem 14} to find $\hat{x}$ which is the least squares solutions.\\
You can also use QR factorization to find the least squares solution as well. This time, use \textbf{Theorem 15} to find the least squares solution.
However, the equation from \textbf{Theorem 14} still works regardless.


\end{document}
