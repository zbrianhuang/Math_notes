\documentclass{article}
\title{Chapter 1}
\author{Brian Huang}
\date{December 1 2023}
\usepackage{mathtools}
\begin{document}
    \maketitle
    \newpage 
    \section*{1.1 System of Linear Equations}

        
    A linear equation in the variables $x_1,x_2,...,x_n$ is an equation that can be written 
    as:
    \begin{center}
        
    
    $a_1x_1+a_2x_2+...+a_nx_n=b$
    
    \end{center}
    where $a_1,a_2,a_n,b$ are real or complex numbers.
    

    \textbf{Example 1:}
    \begin{center}
        $2x_1-5x_2+2=x_1\\
        x_1-5x_2+2=0\\
        x_1-5x_2=-2

    \end{center}

A solution of the system is a list $(s_1,...,s_n)$ of numbers that makes each equation true when $(s_1,...,s_n)$ are plugged into $(x_1,...,x_n)$.
    \\
    \begin{center}
        \fbox{
            \begin{minipage}{15em}
            A system of linear equations has:
            1. No solutions\\
            2. One solution\\
            3. Many solutions\\
            \end{minipage}
        }
    \end{center}
A system of linear equations is said to be consistent if it has one or many solutions, an inconsistent system has no solutions.

\subsection*{Matrix Notation}
The essential information of a linear system can be recorded in a matrix.

\textbf{Example}
\\Given:
    \begin{center}
        $x_1 - x_2 + x_3 = 0$\\
         $2x_2-8x_3=8$\\
         $5x_1-5x_3=10$
    \end{center}

    the coefficents can be aligned in columns into:
    \begin{center}
        
        \begin{bmatrix*}[l]
            1\phantom{-} & \llap{$-$}1 & 1\\
            0 & 2 & \llap{$-$}8\\
            5 & 0 & \llap{$-$}5
    \end{bmatrix*}
    \end{center}    

\end{document}
