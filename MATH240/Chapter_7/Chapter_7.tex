\documentclass{article}

\usepackage{mathtools}
\title{Chapter 7}
\author{Brian Huang}
\date{December 2, 2023}
\begin{document}
\setlength{\parindent}{0pt}%
    \maketitle
    \newpage
    \section*{7.1 Diagonalization of Symmetric Matrices}
    \textbf{Definition:}
    A symmetric matrix is a matrix $A$ such that  $A^T=A$\\
    This means:\\
    $A$ will be square
    The diagonals can be arbitrary but its other entries occur in pairs- on opposite sides of the main diagonal.
\\\\
    \textbf{Example}\\
    Symmetric Matrices:
    \begin{center}
        \[
        \begin{bmatrix*}[r]
             1 & 0\\
             0 & -3
        \end{bmatrix*}
        ,
        \begin{bmatrix*}[r]
            0 & -1 & 0\\
            -1 & 5 & 8\\
            0 & 8 & 7
            

        \end{bmatrix*}
        ,
        \begin{bmatrix*}[r]
            $a$ & $b$ & $c$ \\
            $b$ & $d$ & $e$\\
            $c$ & $e$ & $f$
        \end{bmatrix*}
    
        
        \]

    \end{center}


    \\
    Non-Symmetric Matrices:
    \begin{center}
        \[
        \begin{bmatrix*}[r]
            1 & -3\\
            3 & 0\\
        \end{bmatrix*}
        ,
        \begin{bmatrix*}[r]
            1 & -4 & 0\\
            -6 & 1 & -4\\
            0 & -6 & 1
        \end{bmatrix*}


    \]
    \end{center}
    \\
    \textbf{Theorem}\\
    If $A$ is symmetric, then any two eigenvectors from different eigenspaces are orthogonal.\\

    \textbf{Theorem}
   
    

    \\
    An $n*n$ matrix is orthogonally diagonizable if and only if A is a symmetric matrix.\\

    \textbf{Theorem}
    The Spectral Theorem for Symmetric Matrices: \\
    An $n*n$ symmetric matrix $A$ has the following properties:\\
    1) $A$ has $n$ real eignvalues, counting multiplicites.\\
    2) The dimension of the eigenspace for each eigenvalue $\lambda$ equals the multiplcity of $\lambda$ as a root of the characteristic equation.\\
    3) The eigenspaces are mutually orthogonal.\\
    4) $A$ is orthogonally diagonalizable. 

    \subsection*{Spectral Decomposition}
    If $A = PDP^{-1}$, where the columns of P are orthonormal eigenvectors $u_1,...,u_n$ of $A$ and the 
    corresponding eigenvalues $\lambda_1,....,\lambda_n$ are in the diagonal matrix  $D$.\\
    Then, $P^T = P^{-1}$
    \\
    \\
    \\
    The following equation represents A as a spectral decomposition since it breaks up $A$ into pieces
    determined by the spectrum (eigenvalues). 
    \begin{center}
        $A = \lambda_1 u_1 u_1^T+\lambda_2u_2u_2^T+...+\lambda_nu_nu_n^T$
    \end{center}
    
    \subsection*{Conclusion}
    So basically...\\
    Symmetric matrices are matrices where the pairs on opposite sides are equivalent to each other.
    While the diagonals can be anything. 
    

    \end{document}
