% This is a simple sample document.  For more complicated documents take a look in the exercise tab. Note that everything that comes after a % symbol is treated as comment and ignored when the code is compiled.

\documentclass{article} % \documentclass{} is the first command in any LaTeX code.  It is used to define what kind of document you are creating such as an article or a book, and begins the document preamble
\setlength\parindent{0pt}
\usepackage{amsmath} % \usepackage is a command that allows you to add functionality to your LaTeX code

\title{MATH241 Chapter 11} % Sets article title
\author{Brian Huang} % Sets authors name
\date{December 17, 2023} % Sets date for date compiled

% The preamble ends with the command \begin{document}
\begin{document} % All begin commands must be paired with an end command somewhere
    \maketitle
    \newpage
    \section*{11.1 Cartesian Coordinates in Space} 
    
    Given a point in 3 dimensional space $P$, then there are 3 planes that intersect $P$
    and are perpendicular to the x, y, and z axis.
    \\
    So, $P$ can be associated with an ordered triple of numbers $(x,y,z)$.
    This way of writing $P$ is called the rectangular, or Cartesian, coordinates.
    \\
    \subsection*{Distance}
    The distance between two points, $P$ and $Q$, can be found using the equation:
    \begin{center}
        $|PQ| = \sqrt{(x_1-x_0)^2+(y_1-y_0)^2+(z_1-z_0)^2}$
        
    \end{center}
    Some other notes:
    \begin{center}
        $|PQ|=0$ iff $P=Q$\\
        $|PQ| = |QP|$\\
        $|PQ| \leq |PR|+|RQ|$ for any third point $R$
        
        
    \end{center}
    
\section*{11.2 Vectors in Space}

    \subsection*{Definition: Vector}
    A vector is an ordered triple $(a_1,a_2,a_3)$ of numbers. The numbers $a_1,a_2$, and $a_3$
    are called the components of the vector. The vector $\vec{PQ}$ associated with the directed
    line segement with the inital point $P = (x_0,y_0,z_0)$ and the terminal point $Q = (x_1,y_1,z_2)$
    is $(x_1-x_0,y_1-y_0,z_1-z_0)$

    \subsection*{Definition: Norm}
    The length(norm) of a vector $a = (a_1,a_2,a_3)$ is denoted as $\|a\|$ is defined as:
    \begin{center}
        $\|a\|=\sqrt{a^2_1+a^2_2+a^2_3}$
        
    \end{center}
    A \textbf{Unit Vector} is a vector with a norm of 1.

    Some special unit vectors:\\
    $i = (1,0,0)$\\
    $j = (0,1,0)$\\
    $k = (0,0,1)$\\
    A vector $a=(a_1,a_2,a_3)$ can be written as:
    \begin{center}
        $a = a_1i+a_2j+a_3k$
        
    \end{center}


    \subsection*{Vector Operations}
    Let $a = (a_1,a_2,a_3)$, $b = (b_1,b_2,b_3)$ and $c$ be a scalar.

    \begin{center}
        $a+b = (a_1+b_1,a_2+b_2,a_3+b_3)$\\
        $a-b = (a_1-b_1,a_2-b_2,a_3-b_3)$\\
        $ca=(ca_1,ca_2,ca_3)$
        
    \end{center}
    
    There are four ways to describe a vector:
    1. as $(a_1,a_2,a_3)$, an ordered triple of numbers\\
    2. as $(a_1,a_2,a_3)$, a point in space\\
    3. as a directed line segment with an initial point at $(x_0,y_0,z_0)$ and a terminal point at $(x_0+a_1,y_0+a_2,z_0+a_3)$\\
    4. as $a_1i+a_2j+a_3k$

    \subsection*{Parallel Vectors}
    Two nonzero vectors $a$ and $b$ are parllel iff there is exists a scalar $c$ such that
    $b = ca$.
    
    
    
    
    
    
    
    
    
    
    
    
    
    
    

\end{document} % This is the end of the document
